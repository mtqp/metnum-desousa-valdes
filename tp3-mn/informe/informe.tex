%********************************************************************************************
%								COMANDOS ÚTILES PARA LATEX EN ESTE TP							
%
%	\ : espacio simple
%	\\ : nueva línea
%	\par : va a la línea de abajo y deja sangría
%	\vspace{##tamaño en pt##} o \vspace{\baselineskip} en general:
%								 para dejar un espacio vertical
%	\textbf{text} :text en negrita
%	\textit{text} :text en itálica
%
% GRAFICOS CENTRADOS:
%	\begin{center}
%		\includegraphics[width=\textwidth]{./img/##ruta imagen (no hace falta extension)##}
%	\end{center}
%		--> se pueden agregar atributos como scale por si se hace muy grande
%
% TABLAS CENTRADAS:
%	\begin{center}
%	\begin{tabular}{|c|c|}
%	\hline
%	\ \textbf{Programa} & \textbf{Ticks} \\
%	\hline
%		ASM & 675127609 \\
%	\hline
%	\end{tabular}
%	\end{center}
%
% ALGORITMOS (EN VARIOS LENGUAJES):
% \begin{lstlisting}
%	void sumoDiez(int &num)
%	{
%	    num += 10;
%	}
%	
%	int main()
%	{
% 	   int i;
%	    int numeroAProcesar = 20;
%	    for (i = 0; i < 50; i++)
%	    {
%	        sumoDiez(numeroAProcesar);	//Proceso el numero en cada ciclo
%	    } 
%	    return 0;
%	}
%	\end{lstlisting}
%
% para info sobre todo lo que tiene el package detallado:
% http://en.wikibooks.org/wiki/LaTeX/Source\_Code\_Listings
%
%********************************************************************************************

\documentclass[10pt,a4paper]{article}
\usepackage[utf8]{inputenc} % para poder usar tildes en archivos UTF-8
\usepackage[spanish]{babel} % para que comandos como \today den el resultado en castellano
\usepackage{a4wide} % márgenes un poco más anchos que lo usual
%\usepackage{geometry}

%\usepackage{layout}

%\geometry{
%  includeheadfoot,
%  margin=2.7cm
%}

\usepackage[conEntregas]{caratula}
\usepackage{amssymb}
\usepackage{fancybox}
\usepackage[usenames,dvipsnames]{color}
\usepackage{hyperref}
\usepackage{listings}
\usepackage{clrscode3e}
\usepackage{xcolor}
\usepackage{amsmath}


\hypersetup{
    colorlinks,
    citecolor=black,
    filecolor=black,
    linkcolor=black,
    urlcolor=black
}

\lstdefinestyle{customc}{
  belowcaptionskip=1\baselineskip,
  breaklines=true,
  frame=L,
  xleftmargin=\parindent,
  language=C,
  showstringspaces=false,
  basicstyle=\footnotesize\ttfamily,
  keywordstyle=\bfseries\color{green!40!black},
  commentstyle=\itshape\color{purple!40!black},
  identifierstyle=\color{blue},
  stringstyle=\color{orange},
}

\lstset{escapechar=@,style=customc}

\begin{document}

\titulo{Trabajo Práctico 2}
\subtitulo{Develando la mentira de los megapíxeles [Primera entrega]}

\fecha{\today}

\materia{Métodos Numéricos}
\grupo{Grupo Autodenominado "Los Pichis"}

\integrante{De Sousa Bispo, Germán Edgardo}{359/12}{german\_nba11@hotmail.com}
\integrante{De Sousa Bispo, Mariano Edgardo}{389/08}{marian\_sabianaa@hotmail.com}
\integrante{Valdés Castro, Tobías}{800/12}{tobias.vc@hotmail.com}


\maketitle

\tableofcontents
\newpage

\section*{Introducción}
\addcontentsline{toc}{section}{Introducción}

Luego de haber llevado a cabo el TP1 y el TP2 de métodos númericos, la excelente compañía \textbf{Adobby} se vio interesada en nosotros. En vista de que se presentaron tres puestos libres en la compañia (se dice que tres de sus empleados, un flaquito de anteojos, un colorado y una sabelotodo, partieron en búsqueda de un nuevo trabajo en ``El Que No Debe Ser Nombrado''), los recruiters de la empresa lanzaron un desafío para encontrar los reemplazantes de lo que llamaban ``el trío mágico'', y suplantarlos de esta manera en sus puestos de \textit{Ninja Gurú Jedi Master: The image demosaicing god.}

\par 
El desafío consiste en diseñar, implementar y analizar un algoritmo para resolver el
problema de demosaicing. El mismo consiste en obtener una imagen con información en los 3 canales (colores rojo, verde y azul) para cada pixel, a partir de la información capturada por un sensor de una cámara digital marca \textit{``Fotos Casi en Movimiento''}. Para esto se asume que la información captada por el sensor es
correcta y se trata de inferir los valores de los dos canales faltantes en cada uno de los píxeles de la imagen.
\par 
Con el fin de no tener que invocar artes oscuras para resolver este problema, el desafío plantea trabajar solo sobre el Bayer Array. El mismo consiste en alternar filas de rojo y verde o verde y azul. Cada color no recibe una fracción igual en el area ya que el ojo humano es mas sensitivo a la ``luz verde''. Este tipo de arreglo tiene el doble de elementos verdes que rojos o azules, lo cual produce una imagen que aparenta tener menos ruido y mayor detalle que si se trataran los tres colores por igual. El motivo por el que sucede esto escapa los efectos de este trabajo práctico.
\par 
Como parte del desafío de lo que denominaron \textit{``El torneo de los Tres Programadores''}, \textit{Adobby} pidió que se implementaran varios métodos para poder recrear la imagen a partir de lo que tomó la cámara.


\section{Desarrollo}

\subsection{Implementación}
\subsubsection{Closest Neighbor}
\subsubsection{Linear Interpolation}
\subsubsection{Bilinear Interpolation}
\subsubsection{Directional Interpolation}
\subsubsection{Algoritmo de Malvar, He y Cutler}


\subsubsection{Problemas en la Implementación}

	
\subsection{Experimentación}


\subsubsection{Análisis Cuantitavo}

\subsubsection{Análisis Cualitativo}


\section{Conclusión}


\section{Bibliografía y referencias} %arreglar cuando se termine 

\begin{itemize}
	\item \textbf{STL de C++}: \url{http://en.cppreference.com}.
%	\par Para la función \texttt{rand()}, \url{http://en.cppreference.com/w/cpp/numeric/random/rand}.
%	\par Para la función \texttt{sort()}, \url{http://en.cppreference.com/w/cpp/algorithm/sort}.
%	\item Distribución de \texttt{rand()}?, \url{http://eternallyconfuzzled.com/arts/jsw\_art\_r and.aspx}
%	\item \textbf{Métodos Numéricos:}
%		\par Método de la potencia: Richard BURDEN, Numerical Analysis 9th Ed. Chapter 9 Section 3, p. 576
%		\par \textit{Papers} del TP.
%	\item \textbf{Contador de clocks}: \url{http://www.mcs.anl.gov/\~kazutomo/rdtsc.html}
http://www.cambridgeincolour.com/tutorials/camera-sensors.htm
\end{itemize}


\end{document}
