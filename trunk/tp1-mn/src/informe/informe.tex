%********************************************************************************************
%								COMANDOS ÚTILES PARA LATEX EN ESTE TP							
%
%	\ : espacio simple
%	\\ : nueva línea
%	\par : va a la línea de abajo y deja sangría
%	\vspace{##tamaño en pt##} o \vspace{\baselineskip} en general:
%								 para dejar un espacio vertical
%	\textbf{text} :text en negrita
%	\textit{text} :text en itálica
%
% GRAFICOS CENTRADOS:
%	\begin{center}
%		\includegraphics[width=\textwidth]{./img/##ruta imagen (no hace falta extension)##}
%	\end{center}
%		--> se pueden agregar atributos como scale por si se hace muy grande
%
% TABLAS CENTRADAS:
%	\begin{center}
%	\begin{tabular}{|c|c|}
%	\hline
%	\ \textbf{Programa} & \textbf{Ticks} \\
%	\hline
%		ASM & 675127609 \\
%	\hline
%	\end{tabular}
%	\end{center}
%
% ALGORITMOS (EN VARIOS LENGUAJES):
% \begin{lstlisting}
%	void sumoDiez(int &num)
%	{
%	    num += 10;
%	}
%	
%	int main()
%	{
% 	   int i;
%	    int numeroAProcesar = 20;
%	    for (i = 0; i < 50; i++)
%	    {
%	        sumoDiez(numeroAProcesar);	//Proceso el numero en cada ciclo
%	    } 
%	    return 0;
%	}
%	\end{lstlisting}
%
% para info sobre todo lo que tiene el package detallado:
% http://en.wikibooks.org/wiki/LaTeX/Source\_Code\_Listings
%
%********************************************************************************************

\documentclass[10pt,a4paper]{article}
\usepackage[utf8]{inputenc} % para poder usar tildes en archivos UTF-8
\usepackage[spanish]{babel} % para que comandos como \today den el resultado en castellano
\usepackage{a4wide} % márgenes un poco más anchos que lo usual
\usepackage[conEntregas]{caratula}
\usepackage{amssymb}
\usepackage{fancybox}
\usepackage[usenames,dvipsnames]{color}
\usepackage{hyperref}
\usepackage{listings}
\usepackage{clrscode3e}
\usepackage{xcolor}
\usepackage{amsmath}


\hypersetup{
    colorlinks,
    citecolor=black,
    filecolor=black,
    linkcolor=black,
    urlcolor=black
}

\lstdefinestyle{customc}{
  belowcaptionskip=1\baselineskip,
  breaklines=true,
  frame=L,
  xleftmargin=\parindent,
  language=C,
  showstringspaces=false,
  basicstyle=\footnotesize\ttfamily,
  keywordstyle=\bfseries\color{green!40!black},
  commentstyle=\itshape\color{purple!40!black},
  identifierstyle=\color{blue},
  stringstyle=\color{orange},
}

\lstset{escapechar=@,style=customc}

\begin{document}

\titulo{Trabajo Práctico 1}
\subtitulo{Monkey Island [Primera entrega]}

\fecha{\today}

\materia{Métodos Numéricos}
\grupo{Grupo Autodenominado "Los Pichis"}

\integrante{De Sousa Bispo, Germán Edgardo}{359/12}{german\_nba11@hotmail.com}
\integrante{De Sousa Bispo, Mariano Edgardo}{389/08}{marian\_sabianaa@hotmail.com}
\integrante{Valdés Castro, Tobías}{800/12}{tobias.vc@hotmail.com}


\maketitle

\tableofcontents
\newpage

\section*{Introducción}
\addcontentsline{toc}{section}{Introducción}

El problema que se nos presentó es el del capitán Guybrush Threepwood (un gran pirata) y su barco ``El Pepino Marino''. La existencia de las raras sanguijuelas mutantes ocasiona malestar a nuestro estimado capitán. Las mismas se adhieren al parabrisas de la nave y generan temperatura que pone en peligro la integridad del mismo. La transmisión del calor por parte de las sanguijuelas se produce de manera circular y la temperatura es idéntica para todas ellas. 
\par 
El parabrisas no es invencible: si la temperatura en su punto central supera los 235 grados centígrados, las sanguijuelas ganan y el parabrisas se rompe, dejando al capitán Guybrush Threepwood a merced de las alimañas mutantes. A este punto central se lo denominará \textit{punto crítco}.
\par 
Sin embargo, el parabrisas posee un mecanismo de defensa ante las criaturas. Los bordes emiten temperaturas bajo cero, para permitir el enfriamiento y así vencer a las sanguijuelas. El sistema de refrigeramiento emite -100 grados centígrados en todo el borde del parabrisas, disminuyendo de esta forma la temperatura global de la superficie. 
\par 
A su vez, nuestro querido capitán Threepwood posee un pollo de hule (el arma más temible para las criaturas de los Siete Mares) con el cual puede destruir sanguijuelas. Sin embargo, el uso del "Gran Pollo" es agotador para Guybrush, por lo que debe intentar utilizarlo la menor cantidad de veces posibles para no comprometer el resultado de la misión. 

\vspace{\baselineskip}
\par 
Es nuestro deber, como aspirantes a piratas (informáticos), ayudar al capitán Threepwood para determinar la temperatura en el punto crítico y saber si es necesario utilizar el pollo de hule para matar a las malvadas sanguijuelas. Para ello, llevamos la situación al mundo de la Computación: ya que no podemos representar los infinitos puntos que forman el parabrisas, utilizaremos una discretización que nos permitirá computar la resolución a este problema. 
\par
Gracias a las fórmulas otorgadas por la cátedra para obtener la temperatura en cada punto del parabrisas, calcularla en el punto crítico implica saber además \textit{la temperatura en todos los otros puntos}. Es por esto que encontrar una solución al problema se traduce a \textit{plantear y resolver un sistema lineal en el que las incógnitas son las temperaturas de cada posición discreta del parabrisas}. Es decir, matemáticamente, queremos encontrar una solución a la ecuación

\[ Ax = b \]

donde $x$ representa el vector con todas las temperaturas punto a punto del parabrisas discretizado. Esto nos obliga entonces a pensar qué son la matriz $A$ y el vector $b$ en esta ecuación, lo cual será explicado en la sección \textbf{Desarrollo}. Para resolver este sistema se deberá aplicar \textbf{\textit{Eliminación Gaussiana}} a fin de obtener una matriz triangular superior. Luego, se aplicará un algoritmo de resolución llamado \textbf{\textit{backward substitution}} que se encargará de devolvernos todas las temperaturas de los puntos que hayamos podido discretizar.

\section{Desarrollo}

\subsection{Ideas y pseudocódigo}

\subsubsection{Ideas para la implementación con matriz A cuadrada}

Como vimos en la introducción, resolver este problema es resolver el problema de $Ax = b$ siendo $x$ las temperaturas en todos los puntos que se hayan podido discretizar del parabrisas. Este vector $x$ tendrá un tamaño de $(n+1) \times (m+1)$ siendo $n+1$ la cantidad de filas de esa matriz y $m+1$ la cantidad de columnas. Estos $n$ y $m$ se despejan de las ecuaciones $a = m \times h$ y $b = n \times h$, es decir, $m = \frac{a}{h}$ y $n = \frac{b}{h}$ donde $a$ es el ancho del parabrisas, $b$ el alto, y $h$ siendo la discretización elegida. Veremos ahora qué fue lo que pensamos para determinar $A$ y $b$.

\vspace{\baselineskip}

 Teniendo en cuenta la siguiente fórmula para el cálculo de temperaturas

\[
\frac{\partial^2T(x,y)}{\partial x^{2}}+\frac{\partial^2 T(x,y)}{\partial y^{2}} \ \cong \ \frac{ t_{i-1,j} + t_{i+1,j} - 4t_{i,j} + t_{i,j-1} + t_{i,j+1}}{h^2} = 0,
\]

que los puntos de los bordes tienen temperatura fija -100${}^o$C y que los puntos afectados por el radio de ataque de las sanguijuelas valen otra temperatura fijada $t_{sang}$ en los parámetros del problema, podemos plantear a $b$ como el vector que sigue esta función partida para cada posición ($i,j$) de la matriz discreta del parabrisas:

$$f_{b}(i,j) = \left\{
\begin{array}{c l}
 -100 & si \ la \ posicion  \ p_{i,j} \ es  \ un \ borde  \ del \ parabrisas\\
 t_{sang} & si \ la \ posicion  \ p_{i,j} \ es \ afectada \ por \ una \ sanguijuela \\
 0 & en \ otro \ caso \ en \ el \ cual \ no \ conozca \ la \ temperatura \ en \ p_{i,j}
\end{array}
\right.
$$

De esta forma, el vector $b$ tendrá un tamaño igual a la cantidad de elementos de la matriz que representa al parabrisas, es decir $(n+1) \times (m+1)$, al igual que el vector $x$.

\vspace{\baselineskip}

Nos queda definir entonces qué es la matriz $A$. En principio, necesitamos una matriz de alto $(n+1) \times (m+1)$ ya que si no \textit{no podríamos hacer la multiplicación $Ax$}. Decidimos entonces tener una matriz cuadrada de  $((n+1) \times (m+1)) \times  ((n+1) \times (m+1))$ en la cual cada fila representa un elemento de la matriz discreta del parabrisas. Por ejemplo, la primera fila representaría al elemento (0,0), la segunda al (0,1), la tercera al (0,2), hasta la última al (n-1,n-1). A su vez, hay una misma cantidad de columnas para poder multiplicar esta matriz por $x$.

La idea atrás de esto es que, al hacer $Ax$, tengamos un sistema de ecuaciones que incluya la derivada parcial simplificada, que ya vimos cuando no conocemos la temperatura de un punto en particular, e igualdades directas como la temperatura en los bordes o en un punto afectado. Para lograr esto, esta matriz se va a componer principalmente de \textit{ceros y unos}.

\vspace{\baselineskip}

Este sería un ejemplo de lo que queremos llevar a cabo: si en la primera fila pongo el primer elemento en 1 y el resto en 0, al multiplicar por el vector de temperaturas-incógnitas estaria teniendo en mi ecuación el valor de $x_1$. Como este es un borde, en $b$ tendría, según la función ya presentada, un -100 indicando la temperatura fija adecuada. Luego mi ecuación sería $x_1 = -100$, y esto es justo lo que queremos. Si el ejemplo sirviera para encontrar la siguiente temperatura, me ubicaria en la segunda fila, y como este también es un borde, debería colocar un $1$ en la segunda posición de la fila, así al multiplicar obtendría el $x_2 = -100$. Luego, para cada fila, voy a mirar el elemento que pertenece a la diagonal de la matriz cuadrada ya que allí se encontrará la temperatura a calcular asociada a esa posición en $x$.

En un caso más complicado como en el de temperatura desconocida, deberíamos poner $1$ en cada lugar correspondiente a $t_{i-1,j}$, $t_{i+1,j}$, $t_{i,j-1}$ y $t_{i,j+1}$ involucrado en la ecuación, y un -4 en $t_{i,j}$. Esos valores en $x$ se ubicarán de la siguiente forma: para $t_{i,j}$ miro el elemento perteneciente a la diagonal en esa fila tal como habíamos dicho y coloco allí un -4 ya que ese es su coeficiente asignado en la ecuación de la derivada parcial simplificada. Para $t_{i,j-1}$ y $t_{i,j+1}$, como solo se cambia una columna, estarán una posición menos o más en $x$, por lo tanto puedo poner un 1 a los costados del -4. Luego, los siguientes 1 estarán a $(n+1)$ ($t_{i-1,j}$) o $-(n+1)$ ($t_{i+1,j}$) espacios a los costados del -4, ya que $n+1$ es el ancho de la matriz del parabrisas discreto, y al cambiar una fila ($i-1$ o $i+1$ en las expresiones de $t$) voy a correrme exactamente $n+1$ espacios para la izquierda o la derecha.

\vspace{\baselineskip}

Veamos un ejemplo para una matriz pequeña como un parabrisas de $2 \times 2$ con discretización 1, sin sanguijuelas. Como habíamos dicho, la matriz discretizada del parabrisas será de tamaño $(n+1) \times (m+1)$, luego de $3 \times 3$ en este ejemplo. Así sería la matriz discreta del parabrisas: 

\vspace{\baselineskip}

\[ \left( \begin{array}{ccc}
-100 & -100 & -100 \\
-100 & ??? & -100 \\
-100 & -100 & -100 
\end{array} \right)\] 

\vspace{\baselineskip}

Como los primeros y últimos 4 elementos de la matriz discreta del parabrisas son bordes, hay temperatura -100. En el medio desconocemos la temperatura.

Por otro lado, nuestra matriz $A$ tendrá dimensiones cuadradas de $((n+1) \times (m+1)) \times  ((n+1) \times (m+1))$, luego en este caso, será de $9 \times 9$.

\vspace{\baselineskip}

\[ \left( \begin{array}{ccccccccc}
1 & 0 & 0 & 0 & 0 & 0 & 0 & 0 & 0 \\
0 & 1 & 0 & 0 & 0 & 0 & 0 & 0 & 0 \\
0 & 0 & 1 & 0 & 0 & 0 & 0 & 0 & 0 \\ 
0 & 0 & 0 & 1 & 0 & 0 & 0 & 0 & 0 \\
0 & 1 & 0 & 1 & -4 & 1 & 0 & 1 & 0 \\
0 & 0 & 0 & 0 & 0 & 1 & 0 & 0 & 0 \\ 
0 & 0 & 0 & 0 & 0 & 0 & 1 & 0 & 0 \\
0 & 0 & 0 & 0 & 0 & 0 & 0 & 1 & 0 \\
0 & 0 & 0 & 0 & 0 & 0 & 0 & 0 & 1
\end{array} \right)\] 

\vspace{\baselineskip}

Tal como habíamos dicho, cada fila $i$ de esta matriz representa al elemento $i$ de la matriz del parabrisas (vista como un vector aplanado por filas). En la matriz del parabrisas, los elementos que pudimos completar con -100 representan a los bordes del parabrisas. Siguiendo la descripción que habíamos hecho de la matriz $A$, habrá un 1 en la posición $(i,i)$ para las filas que se correspondan con estos bordes, y el resto de las posiciones serán 0. Esto vale para toda la matriz salvo por el medio, en donde desconocemos la temperatura. Luego para la fila $4$ (empezando a contar desde 0), es decir la fila correspondiente al elemento del medio de la matriz del parabrisas, debemos llenar los casilleros con -4 y 1 según nos convenga para obtener luego de hacer $Ax$ la ecuación de la derivada parcial simplificada.

Los vectores $x$ y $b$ entonces tiene la forma siguiente:

\[ x =  \left( \begin{array}{c}
t_{0,0} \\
t_{0,1} \\
t_{0,2} \\
t_{1,0} \\
t_{1,1} \\
t_{1,2} \\
t_{2,0} \\
t_{2,1} \\
t_{2,2} \\
\end{array} \right), \  b = \left( \begin{array}{c}
-100 \\
-100 \\
-100 \\
-100 \\
0 \\
-100 \\
-100 \\
-100 \\
-100 \\
\end{array} \right)
\] 

Juntando todo nos queda entonces:

\[ Ax = b \]

$$
\left( \begin{array}{ccccccccc}
1 & 0 & 0 & 0 & 0 & 0 & 0 & 0 & 0 \\
0 & 1 & 0 & 0 & 0 & 0 & 0 & 0 & 0 \\
0 & 0 & 1 & 0 & 0 & 0 & 0 & 0 & 0 \\ 
0 & 0 & 0 & 1 & 0 & 0 & 0 & 0 & 0 \\
0 & 1 & 0 & 1 & -4 & 1 & 0 & 1 & 0 \\
0 & 0 & 0 & 0 & 0 & 1 & 0 & 0 & 0 \\ 
0 & 0 & 0 & 0 & 0 & 0 & 1 & 0 & 0 \\
0 & 0 & 0 & 0 & 0 & 0 & 0 & 1 & 0 \\
0 & 0 & 0 & 0 & 0 & 0 & 0 & 0 & 1
\end{array} \right)	\left( \begin{array}{c}
t_{0,0} \\
t_{0,1} \\
t_{0,2} \\
t_{1,0} \\
t_{1,1} \\
t_{1,2} \\
t_{2,0} \\
t_{2,1} \\
t_{2,2} \\
\end{array} \right) = \left( \begin{array}{c}
-100 \\
-100 \\
-100 \\
-100 \\
0 \\
-100 \\
-100 \\
-100 \\
-100 \\
\end{array} \right)
$$

\subsubsection{Ideas para la implementación con matriz A banda}

\subsection{Implementación}

\subsection{Experimentación}

\section{Conclusión}

\section{Apéndice}

\section{Bibliografía y referencias} %arreglar cuando se termine 

\begin{itemize}
	\item \textbf{STL de C++}: \url{http://en.cppreference.com}.
	\par Para la función \texttt{rand()}, \url{http://en.cppreference.com/w/cpp/numeric/random/rand}.
	\par Para la función \texttt{sort()}, \url{http://en.cppreference.com/w/cpp/algorithm/sort}.
	\item Distribución de \texttt{rand()}?, \url{http://eternallyconfuzzled.com/arts/jsw\_art\_rand.aspx}
	\item \textbf{Contador de clocks}: \url{http://www.mcs.anl.gov/\~kazutomo/rdtsc.html}
\end{itemize}


\end{document}
