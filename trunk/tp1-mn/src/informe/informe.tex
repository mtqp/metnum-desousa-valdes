%********************************************************************************************
%								COMANDOS ÚTILES PARA LATEX EN ESTE TP							
%
%	\ : espacio simple
%	\\ : nueva línea
%	\par : va a la línea de abajo y deja sangría
%	\vspace{##tamaño en pt##} o \vspace{\baselineskip} en general:
%								 para dejar un espacio vertical
%	\textbf{text} :text en negrita
%	\textit{text} :text en itálica
%
% GRAFICOS CENTRADOS:
%	\begin{center}
%		\includegraphics[width=\textwidth]{./img/##ruta imagen (no hace falta extension)##}
%	\end{center}
%		--> se pueden agregar atributos como scale por si se hace muy grande
%
% TABLAS CENTRADAS:
%	\begin{center}
%	\begin{tabular}{|c|c|}
%	\hline
%	\ \textbf{Programa} & \textbf{Ticks} \\
%	\hline
%		ASM & 675127609 \\
%	\hline
%	\end{tabular}
%	\end{center}
%
% ALGORITMOS (EN VARIOS LENGUAJES):
% \begin{lstlisting}
%	void sumoDiez(int &num)
%	{
%	    num += 10;
%	}
%	
%	int main()
%	{
% 	   int i;
%	    int numeroAProcesar = 20;
%	    for (i = 0; i < 50; i++)
%	    {
%	        sumoDiez(numeroAProcesar);	//Proceso el numero en cada ciclo
%	    } 
%	    return 0;
%	}
%	\end{lstlisting}
%
% para info sobre todo lo que tiene el package detallado:
% http://en.wikibooks.org/wiki/LaTeX/Source\_Code\_Listings
%
%********************************************************************************************

\documentclass[10pt,a4paper]{article}
\usepackage[utf8]{inputenc} % para poder usar tildes en archivos UTF-8
\usepackage[spanish]{babel} % para que comandos como \today den el resultado en castellano
\usepackage{a4wide} % márgenes un poco más anchos que lo usual
\usepackage[conEntregas]{caratula}
\usepackage{amssymb}
\usepackage{fancybox}
\usepackage[usenames,dvipsnames]{color}
\usepackage{hyperref}
\usepackage{listings}
\usepackage{clrscode3e}
\usepackage{xcolor}
\usepackage{amsmath}


\hypersetup{
    colorlinks,
    citecolor=black,
    filecolor=black,
    linkcolor=black,
    urlcolor=black
}

\lstdefinestyle{customc}{
  belowcaptionskip=1\baselineskip,
  breaklines=true,
  frame=L,
  xleftmargin=\parindent,
  language=C,
  showstringspaces=false,
  basicstyle=\footnotesize\ttfamily,
  keywordstyle=\bfseries\color{green!40!black},
  commentstyle=\itshape\color{purple!40!black},
  identifierstyle=\color{blue},
  stringstyle=\color{orange},
}

\lstset{escapechar=@,style=customc}

\begin{document}

\titulo{Trabajo Práctico 1}
\subtitulo{Monkey Island [Primera entrega]}

\fecha{\today}

\materia{Métodos Numéricos}
\grupo{Grupo Autodenominado "Los Pichis"}

\integrante{De Sousa Bispo, Germán Edgardo}{359/12}{german\_nba11@hotmail.com}
\integrante{De Sousa Bispo, Mariano Edgardo}{389/08}{marian\_sabianaa@hotmail.com}
\integrante{Valdés Castro, Tobías}{800/12}{tobias.vc@hotmail.com}


\maketitle

\tableofcontents
\newpage

\section*{Introducción}
\addcontentsline{toc}{section}{Introducción}

El problema que se nos presentó es el del capitán Guybrush Threepwood (un gran pirata) y su barco ``El Pepino Marino''. La existencia de las raras sanguijuelas mutantes ocasiona malestar a nuestro estimado capitán. Las mismas se adhieren al parabrisas de la nave y generan temperatura que pone en peligro la integridad del mismo. La transmisión del calor por parte de las sanguijuelas se produce de manera circular y la temperatura es idéntica para todas ellas. 
\par 
El parabrisas no es invencible: si la temperatura en su punto central supera los 235 grados centígrados, las sanguijuelas ganan y el parabrisas se rompe, dejando al capitán Guybrush Threepwood a merced de las alimañas mutantes. A este punto central se lo denominará \textit{punto crítco}.
\par 
Sin embargo, el parabrisas posee un mecanismo de defensa ante las criaturas. Los bordes emiten temperaturas bajo cero, para permitir el enfriamiento y así vencer a las sanguijuelas. El sistema de refrigeramiento emite -100 grados centígrados en todo el borde del parabrisas, disminuyendo de esta forma la temperatura global de la superficie. 
\par 
A su vez, nuestro querido capitán Threepwood posee un pollo de hule (el arma más temible para las criaturas de los Siete Mares) con el cual puede destruir sanguijuelas. Sin embargo, el uso del "Gran Pollo" es agotador para Guybrush, por lo que debe intentar utilizarlo la menor cantidad de veces posibles para no comprometer el resultado de la misión. 

\vspace{\baselineskip}
\par 
Es nuestro deber, como aspirantes a piratas (informáticos), ayudar al capitán Threepwood para determinar la temperatura en el punto crítico y saber si es necesario utilizar el pollo de hule para matar a las malvadas sanguijuelas. Para ello, llevamos la situación al mundo de la Computación: ya que no podemos representar los infinitos puntos que forman el parabrisas, utilizaremos una discretización que nos permitirá computar la resolución a este problema. 
\par
Gracias a las fórmulas otorgadas por la cátedra para obtener la temperatura en cada punto del parabrisas, calcularla en el punto crítico implica saber además \textit{la temperatura en todos los otros puntos}. Es por esto que encontrar una solución al problema se traduce a \textit{plantear y resolver un sistema lineal en el que las incógnitas son las temperaturas de cada posición discreta del parabrisas}. Es decir, matemáticamente, queremos encontrar una solución a la ecuación

\[ Ax = b \]

donde $x$ representa el vector con todas las temperaturas punto a punto del parabrisas discretizado. Esto nos obliga entonces a pensar qué son la matriz $A$ y el vector $b$ en esta ecuación, lo cual será explicado en la sección \textbf{Desarrollo}. Para resolver este sistema se deberá aplicar \textbf{\textit{Eliminación Gaussiana}} a fin de obtener una matriz triangular superior. Luego, se aplicará un algoritmo de resolución llamado \textbf{\textit{backward substitution}} que se encargará de devolvernos todas las temperaturas de los puntos que hayamos podido discretizar.

\section{Desarrollo}


\subsection{Ideas y pseudocódigo}

 Teniendo en cuenta la siguiente fórmula para el cálculo de temperaturas

\[
0 \ \cong\ \frac{ t_{i-1,j} + t_{i+1,j} - 4t_{i,j} + t_{i,j-1} + t_{i,j+1}}{h^2}
\]

(con $h$ siendo la discretización elegida), que los puntos de los bordes tienen temperatura fija -100${}^o$C y que los puntos afectados por el radio de ataque de las sanguijuelas valen otra temperatura fijada $t_{sang}$ en los parámetros del problema, podemos plantear a $b$ como el vector que sigue esta función partida para cada posición ($i,j$) de la matriz discreta del parabrisas:

$$f(i,j) = \left\{
\begin{array}{c l}
 -100 & si \ la \ posicion  \ p_{i,j} \ es  \ un \ borde  \ del \ parabrisas\\
 t_{sang} & si \ la \ posicion  \ p_{i,j} \ es \ afectada \ por \ una \ sanguijuela \\
 0 & en \ otro \ caso \ en \ el \ cual \ no \ conozca \ la \ temperatura \ en \ p_{i,j}
\end{array}
\right.
$$

De esta forma, el vector $b$ tendrá un tamaño igual a la cantidad de elementos de la matriz que representa al parabrisas, es decir $(n+1) \times (m+1)$ siendo $n+1$ la cantidad de filas de esa matriz y $m+1$ la cantidad de columnas.

\subsection{Implementación}

\subsection{Experimentación}

\section{Conclusión}

\section{Apéndice}

\section{Bibliografía y referencias} %arreglar cuando se termine 

\begin{itemize}
	\item \textbf{STL de C++}: \url{http://en.cppreference.com}.
	\par Para la función \texttt{rand()}, \url{http://en.cppreference.com/w/cpp/numeric/random/rand}.
	\par Para la función \texttt{sort()}, \url{http://en.cppreference.com/w/cpp/algorithm/sort}.
	\item Distribución de \texttt{rand()}?, \url{http://eternallyconfuzzled.com/arts/jsw\_art\_rand.aspx}
	\item \textbf{Contador de clocks}: \url{http://www.mcs.anl.gov/\~kazutomo/rdtsc.html}
\end{itemize}


\end{document}