%********************************************************************************************
%								COMANDOS ÚTILES PARA LATEX EN ESTE TP							
%
%	\ : espacio simple
%	\\ : nueva línea
%	\par : va a la línea de abajo y deja sangría
%	\vspace{##tamaño en pt##} o \vspace{\baselineskip} en general:
%								 para dejar un espacio vertical
%	\textbf{text} :text en negrita
%	\textit{text} :text en itálica
%
% GRAFICOS CENTRADOS:
%	\begin{center}
%		\includegraphics[width=\textwidth]{./img/##ruta imagen (no hace falta extension)##}
%	\end{center}
%		--> se pueden agregar atributos como scale por si se hace muy grande
%
% TABLAS CENTRADAS:
%	\begin{center}
%	\begin{tabular}{|c|c|}
%	\hline
%	\ \textbf{Programa} & \textbf{Ticks} \\
%	\hline
%		ASM & 675127609 \\
%	\hline
%	\end{tabular}
%	\end{center}
%
% ALGORITMOS (EN VARIOS LENGUAJES):
% \begin{lstlisting}
%	void sumoDiez(int &num)
%	{
%	    num += 10;
%	}
%	
%	int main()
%	{
% 	   int i;
%	    int numeroAProcesar = 20;
%	    for (i = 0; i < 50; i++)
%	    {
%	        sumoDiez(numeroAProcesar);	//Proceso el numero en cada ciclo
%	    } 
%	    return 0;
%	}
%	\end{lstlisting}
%
% para info sobre todo lo que tiene el package detallado:
% http://en.wikibooks.org/wiki/LaTeX/Source\_Code\_Listings
%
%********************************************************************************************

\documentclass[10pt,a4paper]{article}
\usepackage[utf8]{inputenc} % para poder usar tildes en archivos UTF-8
\usepackage[spanish]{babel} % para que comandos como \today den el resultado en castellano
\usepackage{a4wide} % márgenes un poco más anchos que lo usual
%\usepackage{geometry}

%\usepackage{layout}

%\geometry{
%  includeheadfoot,
%  margin=2.7cm
%}

\usepackage[conEntregas]{caratula}
\usepackage{amssymb}
\usepackage{fancybox}
\usepackage[usenames,dvipsnames]{color}
\usepackage{hyperref}
\usepackage{listings}
\usepackage{clrscode3e}
\usepackage{xcolor}
\usepackage{amsmath}


\hypersetup{
    colorlinks,
    citecolor=black,
    filecolor=black,
    linkcolor=black,
    urlcolor=black
}

\lstdefinestyle{customc}{
  belowcaptionskip=1\baselineskip,
  breaklines=true,
  frame=L,
  xleftmargin=\parindent,
  language=C,
  showstringspaces=false,
  basicstyle=\footnotesize\ttfamily,
  keywordstyle=\bfseries\color{green!40!black},
  commentstyle=\itshape\color{purple!40!black},
  identifierstyle=\color{blue},
  stringstyle=\color{orange},
}

\lstset{escapechar=@,style=customc}

\begin{document}

\titulo{Trabajo Práctico 2}
\subtitulo{Tirate un qué, tirate un ranking... [Primera entrega]}

\fecha{\today}

\materia{Métodos Numéricos}
\grupo{Grupo Autodenominado "Los Pichis"}

\integrante{De Sousa Bispo, Germán Edgardo}{359/12}{german\_nba11@hotmail.com}
\integrante{De Sousa Bispo, Mariano Edgardo}{389/08}{marian\_sabianaa@hotmail.com}
\integrante{Valdés Castro, Tobías}{800/12}{tobias.vc@hotmail.com}


\maketitle

\tableofcontents
\newpage

\section*{Introducción}
\addcontentsline{toc}{section}{Introducción}


\section{Desarrollo}
\subsection{Ideas sobre la Implementación}

\subsubsection{Características generales del problema}


\subsection{Implementación}
\subsubsection{Implementación con matriz Column Row Sparse (CRS)}

\subsubsection{Problemas en la Implementación}


\subsection{Experimentación}
\subsubsection{Convergencia de PageRank}

\subsubsection{Convergencia de HITS}

\subsubsection{Tiempo de Cómputo}	

\subsubsection{Análisis Cualitativo}

\subsubsection{Casos de Ejemplo}
	
	
\section{Conclusión}


\section{Bibliografía y referencias} %arreglar cuando se termine 

\begin{itemize}
	\item \textbf{STL de C++}: \url{http://en.cppreference.com}.
%	\par Para la función \texttt{rand()}, \url{http://en.cppreference.com/w/cpp/numeric/random/rand}.
%	\par Para la función \texttt{sort()}, \url{http://en.cppreference.com/w/cpp/algorithm/sort}.
%	\item Distribución de \texttt{rand()}?, \url{http://eternallyconfuzzled.com/arts/jsw\_art\_r and.aspx}
	\item \textbf{Funciones de Métodos Numéricos:}
		\par Eliminación Gaussiana y \textit{backward substitution}: Richard BURDEN, Numerical Analysis 9th Ed. Chapter 6
	\item \textbf{Contador de clocks}: \url{http://www.mcs.anl.gov/\~kazutomo/rdtsc.html}
\end{itemize}


\end{document}
